\documentclass[11pt, letterpaper]{article} %Sets the default text size to 11pt and class to article
\usepackage{times}
\usepackage{nopageno}
\usepackage{titlesec}
\usepackage{titling}
\usepackage[margin=0.5in]{geometry}
\usepackage{hyperref}

\titleformat{\section}
{\Large\bfseries\uppercase}  %this argument for the formatting
{}  %this argument is for numbering
{0em}  %distance between the number and text
{}[\titlerule]  %after the gap between the number and title

\titlespacing{\section}
{0em}    %left margin
{0em}    %space between
{0em}    %space afterward

\titleformat{\subsection}[runin]
{\bfseries\large}
{}
{0em}
{}

\titlespacing{\subsection}
{0em}    %left margin
{0em}    %space between
{0.5em}  %space afterward

%%%%%%%%%%%%%%%%%%%%%%%%%%%%%%% Title %%%%%%%%%%%%%%%%%%%%%%%%%%%%%%%%

\renewcommand{\maketitle}{
\begin{center}
{\Huge\bfseries
\theauthor}

266 Union Ave, Apt. 2, New Rochelle, NY 10801

\href{mailto:xinkaili@buffalo.edu}{xinkaili@buffalo.edu} $\bullet$ 646-309-3660

\end{center}
}

%%%%%%%%%%%%%%%%%%%%%%%%%%%%%%%% Begin %%%%%%%%%%%%%%%%%%%%%%%%%%%%%%%

\begin{document}

\title{Resume} 
\author{Xinkai Lin}

\maketitle

%%%%%%%%%%%%%%%%%%%%%%%%%%%%%% Education %%%%%%%%%%%%%%%%%%%%%%%%%%%%%

\section{Education}
\subsection{University at Buffalo, The state University of New York}
\hspace*{\fill}May 2019

\noindent Bachelor of Science in Computer Science 
\hspace*{\fill}GPA: 3.49/4.0\\

%%%%%%%%%%%%%%%%%%%%%%%%%%%%%%% SKILLS %%%%%%%%%%%%%%%%%%%%%%%%%%%%%%%%

\section{SkillS}
\subsection{Programming Skills}
: Java, Python, C/C++, MySQL, HTML, CSS, JavaScript, Vue, Electron

\subsection{Programs}
: Microsoft Office, Eclipse, Version Control(Git), Unity

\subsection{Language}
: Fluent in Mandarin Chinese\\

%%%%%%%%%%%%%%%%%%%%%%%% Professional Experience %%%%%%%%%%%%%%%%%%%%%

\section{Professional Experience}
\subsection{Web Applications for Movies}
-- MySQL, EC2, Apache Tomcat, Bootstrap, jQuery, JBDC
\hspace*{\fill}Nov 2018 -- Jan 2019
\vspace{-0.8em}
\begin{itemize}
	\setlength\itemsep{-0.5em}
	\item Launched a free Amazon Web Services EC2 instance and setup MySQL and Apache Tomcat
	\item Created a MySQL Database with several tables for the movie list and its information, and insert preparatory movie data into the table
	\item Built the movie website which contains a login and main page with capacity of searching and browsing used HTML, CSS, JavaScript and imports Bootstrap and jQuery
	\item Created Java Servlet which handles the login request and movie information request from frontend and returns the result in the JSON format
\end{itemize}
\vspace{-0.8em}

\subsection{Parking Assistant}
-- Realtime Embed System, Arduino Adafruit METRO M0 Express
\hspace*{\fill}Sep 2018 -- Dec 2018
\vspace{-0.8em}
\begin{itemize}
	\setlength\itemsep{-0.5em}
	\item Implement four ultrasonic sensors to calculated the distances between two cars and used it as references to find out a best routine for parallel parking
	\item Equipped one red LED and one buzzer as part of the alarm system and activated if the distance of the object car is too close
	\item Implemented a Liquid Crystal Display for giving instructions on how to parked the car in parallel parking
\end{itemize}
\vspace{-0.8em}

\subsection{Geek Fantasy(2D Unity Game)}
-- Unity, C\#, AI
\hspace*{\fill}Jan 2018 -- Mar 2018
\vspace{-0.8em}
\begin{itemize}
	\setlength\itemsep{-0.5em}
	\item Created a Two-dimensional Role Play Game by Unity included one player and approximately thirty enemy's game objects in eight different play scenes
	\item Implemented player’s movement and attack script written by C\#, so that player is able to control the character to move in four directions and attack enemy in their desire
	\item Implemented enemy’s AI script included enemy’s automatic tracing and attack ability, enemy would automatically trace and attacked player when their distance is within certain range
	\item Applied collision to all game objects, so that they are able to detected the intersection of two or more objects
\end{itemize}
\vspace{-0.8em}

\subsection{Interpreter}
-- SML, Python
\hspace*{\fill}Jan 2018 -- Mar 2018
\vspace{-0.8em}
\begin{itemize}
	\setlength\itemsep{-0.5em}
	\item Built an interpreter which is able to read and perform instructions from input file and return the result into the output file in both Python and SML
	\item Implemented basic computations as well as adding support for immutable variables and structures for expressing scope, finally perform function call
\end{itemize}
\vspace{-0.8em}

\subsection{Escape-time Fractals}
-- Eclipse, Java
\hspace*{\fill}Jan 2017 -- Mar 2017
\vspace{-0.8em}
\begin{itemize}
	\setlength\itemsep{-0.5em}
	\item Created four Escape-time Fractals by designed a GUI and implemented an algorithm to calculated the formulas used Java in Eclipse
	\item Implement functions included zoom-in, resized fractal panels, drop down menu with multiple different options
	\item Collaborated with other three undergraduates' students in minimum of two meeting per week throughout the semester
\end{itemize}

%%%%%%%%%%%%%%%%%%%%%%%%% Research Experience %%%%%%%%%%%%%%%%%%%%%%%%

\section{Research Experience}
\subsection{Buffalo Botanical Garden}
(Research Assistant) -- Visual Studio, HTML 
\hspace*{\fill}Jan 2018 -- Mar 2018
\vspace{-0.8em}
\begin{itemize}
	\setlength\itemsep{-0.5em}
	\item Categorized two list of elements into four separate groups of fields
	\item Redesigned the structure by centered all the information, displayed in full screen mode and changed background color to beautify the edit page
\end{itemize}

\end{document}